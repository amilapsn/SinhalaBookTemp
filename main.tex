\documentclass[10pt]{book}
\usepackage{sinhala}
\usepackage{csquotes}

\def\ord{\mathrm{ord}}
\newenvironment{proof}[1][සාධනය]{%
  \par\noindent\textsl{#1.}\ }%   ← opening code
  {\hfill$\square$\par}%          ← closing code

\title{\bf ක්වො​​න්ට​​ම් ප​​රිග​ණක විද්‍යාව හා ග​​ණිත​ය}
\author{​ටෙ​ඕ අ​​න්​ත්‍රෝ}


\begin{document}                        
\frontmatter                            
\maketitle                              
\tableofcontents                        
\listoffigures
\listoftables
\mainmatter  
\chapter{පූර්වාවශ්‍යතා}
\section{සමූහ මත වූ සාධාරණත්​ව}
\begin{definition}\label{def:group}
\begin{align*}
G \times G \to G\\
(a, b) \mapsto a\cdot b 
\end{align*}
යන ද්විමය කර්මය උපාධාර කොටගත් $G$ කුලකයක් \textbf{සමූහයකි}. මෙහි,
\begin{itemize}
\item[$\bullet$] $\cdot$ කර්ම​ය සාංගමික වේ. එනම්, $\forall a, b, c \in G,\ (a \cdot b)\cdot c = a\cdot (b \cdot c)$.
\item[$\bullet$] $G$ හි අනන්‍යතා අවයවයක් පවතියි. එනම්, $\exists e \in G,\ \forall a \in G,\ e \cdot a = a \cdot e = a$.
\item[$\bullet$] සෑම $a \in G$ සඳහාම එහි ප්‍රතිලෝ​ම අවයවයක් පවතියි. එනම්, $\forall a\in G,\ \exists a^{-1} \in G,\ a \cdot a^{-1} = a^{-1} \cdot a = e$.
\end{itemize}
\end{definition}
ඉහත නිර්වචනය~\ref{def:group} හි දක්වා ඇති $(G, \cdot)$ හි $\cdot$ කර්මය සාංගමික හා $G$ හි අනන්‍යතා අවයවයක් පැවත, ප්‍රතිලෝ​ම අවයවයක් නොපවතියි නම් එය \textbf{ඒකාභයක්} ලෙස හැඳින්වේ.
\begin{example}
$(\mathbb{Z}, +)$ සමූහය. $e = 0,\ a^{-1} = -a$.
\end{example}
\begin{example}\label{ex:group-non-abelian}
$(\mathfrak{S}_3, \circ)$ හි $\circ$ ශ්‍රිත සංයුතිය වන සමූහය. මෙහි $\mathfrak{S}_3 = \{f:\{1, 2, 3\} \to \{1, 2, 3\} \text{ සමක්ෂේප​ණය}\}$.
\end{example}

\begin{definition}
$\forall a, b \in G,\ a \cdot b = b \cdot a$ වූ $(G, \cdot)$ සමූහයක් \textbf{ඇබේලියානු} සමූහයක් ලෙස හැඳින්වේ.
\end{definition}

ඉහත උදාහරණය~\ref{ex:group-non-abelian} හි සමූහය ඇබේලියානු නොවේ.

\begin{definition}
$(G, \cdot)$ සමූහයක් හා $x\in G$ සඳහා, $x$ හි \textbf{ගණය} යනු $\#\{x^n \in G : n \in \mathbb{Z}\}$ වේ. මෙහි $x^n = x\cdot x\cdots x$ ($n$ වතාවක්) ලෙස අර්ථ දක්වයි. වඩා ප්‍රත්‍යක්ෂව $\min \{k\ge 1: x^k = e\}$ වන $k$ හි අගය $x$ හි ගණය වේ.
\end{definition}

\begin{example}
$(\mathbb{Z}/5\mathbb{Z}, +)$ සලකන්න. මෙම සමූහය සඳහා $\mathbb{Z}$ කුලකයට \enquote{$n \sim m \leftrightarrow n-m$~$5$හි ගුණාකාරයක් ​වේ} ය​න තුල්‍යතා සම්බන්ධය පනවනු ලැබේ. නිදසුනක් ලෙස $2 \sim 7 \sim 12$ වේ. එය $2\equiv 7 \equiv 12 \pmod{5}$ ලෙස ද අංකනය කළ හැක.  යුක්ලිඩියානු විභාජනයෙන්, ඕනෑම $n = 5k + r$ ($n, k \in \mathbb{Z},\ 0\le r < 5$) ලෙස දැක්විය හැක. මෙය $\overline{n} \equiv \overline{r}$ ලෙස ද අංකනය කළ හැක. $\overline{n}$ මඟින් $n$ හි තුල්‍යතා පන්තිය දක්වයි. මෙම අංකනය අනුගමනය කරමින්, පහත පරිදි සුළු කිරීම් සිදු කළ හැක: $\overline{2} + \overline{6} = \overline{2+6} = \overline{8} = \overline{3}$. $(\mathbb{Z}/5\mathbb{Z}, \times)$ සමූහය සඳහා ද එපරිදි ම සුළු කිරීම් සිදු කළ හැක. එහිදී, $\overline{2} \times \overline{3} = \overline{6} = \overline{1}$ වේ.

දැන්, $(\mathbb{Z}/5\mathbb{Z}, +)$ සලකන්න. එම සමූහයේ $\overline{2}$ හි ගණය සෙවීමට පහත පියවර අනුගමනය කළ හැක:
\begin{align*}
\overline{2}^1 &{}                    &&{}              &&&= \overline{2}\\
\overline{2}^2 &=\overline{2+2}       &&{}              &&&= \overline{4}\\
\overline{2}^3 &=\overline{2+2+2}     &&= \overline{6}  &&&= \overline{1} \\
\overline{2}^4 &=\overline{2+2+2+2}   &&= \overline{8}  &&&= \overline{3} \\
\overline{2}^5 &=\overline{2+2+2+2+2} &&= \overline{10} &&&= \overline{0}\\
\end{align*}
එනයින්, $(\mathbb{Z}/5\mathbb{Z}, +)$ සමූහයේ $2$ හි ගණය $5$ වේ. මෙය $\ord(2) = 5$ ලෙස ද අංකනය කළ හැක.

දැන් $\mathbb{Z}/5\mathbb{Z}$ පාදක කොටගෙන ගුණ්‍යත සමූහය ව්‍යුත්පන්න කිරීම සැලකූ වි​ට අවධානය යොමු කළ යුතු කරුණක් වන්නේ එම කුලක​යේ ඇත්තේ $5$හි මාපාංකානුකූල ​ව ප්‍රතිලෝමී අවයවයන් පමණක් බවයි. එනම්, $(\mathbb{Z}/5\mathbb{Z})^\times = \{x \in \mathbb{Z}/5\mathbb{Z}: \exists y \in \mathbb{Z}/5\mathbb{Z},\ x \cdot y = 1\}$ ලෙස ගුණ්‍යත කුලකය අර්ථ දැක්වෙයි.  $2 \in (\mathbb{Z}/5\mathbb{Z})^\times$ මක්නිසාදයත්, $2 \cdot 3 = 6 \equiv 1 \pmod{5}$.

දැන්, $((\mathbb{Z}/5\mathbb{Z})^\times, \times)$ සමූහයේ $\overline{2}$ හි ගණය සෙවීමට පහත පියවර අනුගමනය කළ හැක:
\begin{align*}
\overline{2}^1 &{}                                   &&{}               &&&= \overline{2}\\
\overline{2}^2 &=\overline{2\times2}                 &&{}               &&&= \overline{4}\\
\overline{2}^3 &=\overline{2\times2\times2}          &&= \overline{8}   &&&= \overline{3} \\
\overline{2}^4 &=\overline{2\times2\times2\times2}   &&= \overline{16}  &&&= \overline{1} \\
\end{align*}
එනයින්, $((\mathbb{Z}/5\mathbb{Z})^\times, \times)$ සමූහයේ $\ord(2) = 4$.
අතිරේක වශයෙන්, $((\mathbb{Z}/5\mathbb{Z})^\times, \times)$ සමූහයේ $2$ ට එම සමූහය තුළ අත් කර ගත හැකි උපරිම ගණය ඇති බැවින්, $2$ එම සමූහයේ ජනකයක් ලෙස හඳුන්වා දිය හැක. මෙය පසුව අර්ථ දක්වනු ලැබේ.
\end{example}

\begin{theorem}[{}ලග්‍රේන්ජ්]
\label{thm:lagrange}
සමූහ $(G, \cdot)$ හි $\forall x \in G$ සඳහා, $\ord(x) \mid |G|$.
\end{theorem}
\begin{proof}
මඟහරින ලදී.
\end{proof}
ප්‍රමේයය~\ref{thm:lagrange} උපයෝගී කොටගෙන ගණ ගණනය පහසු කර ගත හැක. උදාහරණයක් ලෙස, $G = (\mathbb{Z}/15\mathbb{Z}, +)$ සමූහයේ $2$ හි ගණය සඳහා $|G| = 15$හි සාධක ව​න
$1,3,5,15$ යන අගයන් පමණක් පරික්ෂා කිරීම ප්‍රමාණවත් වේ.

\begin{lemma}
පූර්ණ සාධාරණත්වයෙන්, ඕනෑම $n \ge 2$ සඳහා $(\mathbb{Z}/n\mathbb{Z}, +)$ සමූහයක් වේ. මෙහි, $\mathbb{Z}/n\mathbb{Z}$ කුලක​ය යනු $k\sim k' \leftrightarrow k-k'$ $n$හි ගුණාකාරයක් ​වේ යන සම්බන්ධයෙන් ජනිත වූ තුල්‍යතා පන්ති කුලක​ය වන අතර $\overline{k} + \overline{k'} = \overline{k+k'}$ ලෙස අර්ථ දක්වෙයි.
\end{lemma}
\begin{proof}
මඟහරින ලදී.
\end{proof}

එපරිදි ම ගුණ්‍යතා නීතියක් ද $\overline{k} \times \overline{k'} = \overline{k\times k'}$ ලෙස අර්ථ දැක්විය හැකිය. 

\begin{example}
$((\mathbb{Z}/n\mathbb{Z})^\times, \times)$ සමූහයේ ප්‍රතිලෝමී අවයවයන් මොනවා ද? සරල නිදසුනක් ලෙස $(\mathbb{Z}/12\mathbb{Z})^\times$ සලකන්න. පැහැදිළිව $0 \notin (\mathbb{Z}/12\mathbb{Z})^\times$ මක්නිසාදයත් $\forall n \in \mathbb{Z}/12\mathbb{Z},\ 0 \times n = 0 \neq 1$. ($1$, $((\mathbb{Z}/12\mathbb{Z})^\times, \times)$ සමූහයේ අනන්‍යතා අවයවය වේ). $1\times1 \equiv 1 \pmod{12}$ බැවින්, $1$, $(\mathbb{Z}/12\mathbb{Z})^\times$හි ප්‍රතිලෝමී අවයවයක් වේ. $2 * k \equiv 1 \pmod{12}$ වන පරිදි $k \in \mathbb{Z}/12\mathbb{Z}$ නොමැති බැවින් $2$, $(\mathbb{Z}/12\mathbb{Z})^\times$ හි ප්‍රතිලෝමී අවයවයක් නොවේ. එපරිදිම $3, 4, 6, 8, 9, 10$ ද ප්‍රතිලෝමී අවයවයන් නොවන බව පෙන්විය හැක. $5\times5 \equiv 1\pmod{12}$ වන බැවින් $5$ ප්‍රතිලෝමී අවයවයක් වේ. එපරිදි, අනෙක් ප්‍රතිලෝමී අවයවයන් $7, 11$ බව පෙන්විය හැකිය. ඉහත දී $9$ ප්‍රතිලෝමී අවයවයක් නොවන්නේ $9k = 12m + 1$ වන පරිදි $m, k$ නිඛිල දෙකක් නොපවතින බැවිනි. එසේ වන්නේ $9k - 12m = 3(3k - 4m)$ යන්න $3$හි ගුණාකාරයක් වන බැවිනි. $8$ හා $10$ සඳහා ද ඉහත ආකාරයෙන් ප්‍රතිලෝමී නොවන බවට සාධනය කළ හැකිය.
\end{example}

\begin{proposition}
$\overline{k}$, $\mathbb{Z}/n\mathbb{Z}$ හි ගුණ්‍යතව ප්‍රතිලෝමී වන්නේ $\gcd(k, n) = 1$ නම් හා නම්ම පමණි.
\end{proposition}
\begin{proof}
\begin{align*}
\overline{k} \text{ ප්‍රතිලෝමී වේ} &\Leftrightarrow \exists \overline{k'},\ \overline{k}\overline{k'} = \overline{1}\\
&\Leftrightarrow \exists k', m \in \mathbb{Z},\ kk' = 1 + mn\\
&\Leftrightarrow \exists k', m \in \mathbb{Z},\ kk' + (-m)n = 1\\
&\Leftrightarrow \gcd(k, n) = 1.
\end{align*}
\end{proof}
ඉ​​හත අ​වසාන පිය​වර​​ \textbf{බේ​සෝ} නීතිය ලෙ​​ස ද හැඳින්වේ.
RSA කේ​තන​​ය ට ප​හත ප්‍රස්තුත​​ය වැද​ගත් වේ.
\begin{proposition}
\begin{align*}
|(\mathbb{Z}/n\mathbb{Z})^\times| &= \#\{k:1\le k \le n,\ \gcd(k, n) = 1\}\\
                                  &= \phi(n).
\end{align*}
\end{proposition}
\begin{proof}
ස​රල සාධ​නය​​කි.
\end{proof}
ඉ​හත $\phi(n)$ ශ්‍රිත​​ය, ඔ​​යිල​​ර් මුළ​​ස ශ්‍රිත​​ය ලෙ​​ස ද හැඳින්වේ.

\begin{theorem}[{}චීන ශේ​​ෂ ප්‍රමේ​යය-{}චීශේෂ]
$n = p_1^{\alpha_1}\cdots p_r^{\alpha_r}$ ($p_1, \ldots, p_r$ අ​ගය​යන් ප්‍රභින්න ප්‍රථ​මක සංඛ්‍යා වේ) වේ න​​ම් 
\[
\mathbb{Z}/n\mathbb{Z} \overset{f}{\cong} \mathbb{Z}/p_1^{\alpha_1}\times\ldots\times\mathbb{Z}/p_r^{\alpha_r}.
\]
\end{theorem}

\end{document}  